\documentclass[10pt,a4paper]{article}			%format de page
\usepackage[utf8]{inputenc}						%encodage avec caractères accentués
\usepackage[T1]{fontenc}						%encodage
\usepackage[english,frenchb]{babel}						%typographie française
\usepackage{fancyhdr}
\usepackage{lastpage}
\usepackage{graphicx}

% Mise en page
\usepackage{geometry}
\geometry{a4paper,twoside,left=2.5cm,right=2.5cm,top=2.8cm,bottom=2.8cm} %taille et marges
\pagestyle{fancy}								%style de page {fancy(package), empty, plain, headings}
\renewcommand{\headrulewidth}{0pt}				%épaisseur du trait de tête de page (0.4pt)
\fancyhead[L]{}									%pied de page gauche
\fancyhead[C]{}									%pied de page centre
\fancyhead[R]{}									%pied de page droit
\renewcommand{\footrulewidth}{0pt}				%épaisseur du trait du pied de page (0.4pt)
\fancyfoot[L]{}									%pied de page gauche
\fancyfoot[C]{}									%pied de page centrce
\fancyfoot[R]{\thepage{}/\pageref{LastPage}}	%pied de page droit

\thispagestyle{empty}				% Page de garde vide

% Infos générales
\title{Symmetry Detection}
\author{Chantal DING, Chloé MACUR}
\begin{document}

\selectlanguage{english}

\thispagestyle{empty}							%page de garde vide
%\begin{document}
	\begin{center}
		\hfill
		Chantal \bsc{Ding}
		\hfill \hfill
		Chloé \bsc{Macur}
		\hfill ~
		\par
		\noindent
		\\
		\vspace{0.7cm}
		\textit{December 2013}
		\vfill\vfill
		\Huge
		\begin{tabular}{c}
			\hline
			%Projet\\
			{\Large{\textsc{Symmetry Detection}}}\\
			~~~A Planar-Reflective Symmetry\\ Transform for 3D Shapes~~~\\
			\hline
		\end{tabular}
		\large
		\vfill
		 Project report\\
		 INF555 -- Digital Representation and Analysis of Shapes
	\end{center}
	\vfill
	\begin{flushright}
	\includegraphics[scale=0.1]{img/logo_x_h.jpg}
	\end{flushright}
	\newpage 
\newpage
\tableofcontents
\newpage

\textbf{CE QU'ON EST SUPPOSEES FAIRE}: \\
A short (4-6) page report about your work.
The main idea behind doing a project is to expose you to current research in geometric modeling and shape analysis. Therefore, you should try to do something that either involves a recent technique or analyze a theoretical result. Your project should include some amount of independent work, either by implementing a technique and showing its performance on some examples not included by the authors, or by doing some independent theoretical analysis.

	
	\section*{Introduction}
\addcontentsline{toc}{section}{Introduction}

	\begin{description}	
  \item[Symmetry everywhere]:
  
	A lot of 3D shapes, whether natural or man-made, present some kind of symmetry that can be really useful in 3D geometry.  Indeed symmetries allow certain economy, especially in digital representation, but they are also involved in pattern recognition.\\ ... DEVELOP \\
  
  \item[State of the art]:
   \begin{itemize}
  \item  1st  paper
  \item paper we chose
  \end{itemize}
  \end{description}

	\section{Our work}
	
	\subsection{Choice of the paper:  A Planar-Reflective Symmetry Transform for 3D Shapes}
	
	bla bla intéressant bla bla Monte Carlo bla bla évite brute force, sélection intelligente selon l'énergie de la fonction
	
	\subsection{Work produced}
	
	a word about the code, the structure (classes, ..) and comments about the (strategic) choices
	
	\begin{description}	
  \item[LoadImage]: Used to read the pictures and convert them to a better format
 \item[Sampling]: The sampling is performed according to the energy in the function f, allowing us
to focus effort on computations that will contribute to the final answer
 \item[geometry classes (Droite, Point)]: Used to represent lines and points, find the bisection line ...
 \item[PRST]: Computes the PRST itself
 \item[BasicFrame]: Shows the points/lines, mostly for a debug purpose
 \end{description}
	  
  	\section{Comments on the paper}

	\subsection{Adjustments for 2D}
	
the	polar issue, etc...
  	
  	\subsection{Choice of the 1/N factor in the calculus of D}
  	
  	explain that it comes out of the blue ...

	\section*{Conclusion}
\addcontentsline{toc}{section}{Conclusion}
		
\end{document}
