\documentclass[10pt,a4paper]{article}			%format de page
\usepackage[utf8]{inputenc}						%encodage avec caractères accentués
\usepackage[T1]{fontenc}						%encodage
\usepackage[english,frenchb]{babel}						%typographie française
\usepackage{fancyhdr}
\usepackage{lastpage}
\usepackage{graphicx}

% Mise en page
\usepackage{geometry}
\geometry{a4paper,twoside,left=2.5cm,right=2.5cm,top=2.8cm,bottom=2.8cm} %taille et marges
\pagestyle{fancy}								%style de page {fancy(package), empty, plain, headings}
\renewcommand{\headrulewidth}{0pt}				%épaisseur du trait de tête de page (0.4pt)
\fancyhead[L]{}									%pied de page gauche
\fancyhead[C]{}									%pied de page centre
\fancyhead[R]{}									%pied de page droit
\renewcommand{\footrulewidth}{0pt}				%épaisseur du trait du pied de page (0.4pt)
\fancyfoot[L]{}									%pied de page gauche
\fancyfoot[C]{}									%pied de page centrce
\fancyfoot[R]{\thepage{}/\pageref{LastPage}}	%pied de page droit

\thispagestyle{empty}				% Page de garde vide

% Infos générales
\title{Symmetry Detection}
\author{Chantal DING, Chloé MACUR}
\begin{document}

\selectlanguage{english}

\thispagestyle{empty}							%page de garde vide
%\begin{document}
	\begin{center}
		\hfill
		Chantal \bsc{Ding}
		\hfill \hfill
		Chloé \bsc{Macur}
		\hfill ~
		\par
		\noindent
		\\
		\vspace{0.7cm}
		\textit{December 2013}
		\vfill\vfill
		\Huge
		\begin{tabular}{c}
			\hline
			%Projet\\
			{\Large{\textsc{Symmetry Detection}}}\\
			~~~A Planar-Reflective Symmetry\\ Transform for 3D Shapes~~~\\
			\hline
		\end{tabular}
		\large
		\vfill
		 Project report\\
		 INF555 -- Digital Representation and Analysis of Shapes
	\end{center}
	\vfill
	\begin{flushright}
	\includegraphics[scale=0.1]{img/logo_x_h.jpg}
	\end{flushright}
	\newpage 
\newpage
\tableofcontents

\newpage
	
	\section*{Introduction}
\addcontentsline{toc}{section}{Introduction}
  
	A lot of 3D shapes, whether natural or man-made, present some kind of symmetry that can be really useful in computer vision and 3D geometry.  Indeed symmetries allow certain economy, especially in digital representation, but they are also involved in pattern recognition or geometry completion. Thus, numerous methods are used to detect symmetries, partly because of the diversity of datas (point clouds, polygon meshes, NURBS, patches, etc.).  
 
	\section{A Planar-Reflective Symmetry Transform for 3D Shapes}
	
		%TODO delete before saving
	\textit{· 1 page description of the method you're implementing. (A Planar-Reflective Symmetry Transform for 3D Shapes paper) It goes without saying that you shouldn't copy text from the paper, but rather describe in your own words what the method is trying to do and how it does it. \\}
	
	We chose to try to implement the planar reflective symmetry transform (PRST) described by Podolak et al. ~\cite{Podolak:2006:APS}.
	
	Monte Carlo évite brute force, sélection intelligente selon l'énergie de la fonction
	
	\section{Our implementation}
	
	%TODO delete before saving
	\textit{· 2 page description of your implementation. Here you should describe the main building blocks of your implementation. We are especially interested in: whether you had any problems, whether there were things not mentioned in the paper that you had to discover or derive yourself (be very explicit about your own work!), whether you used any external libraries, etc. What we're not interested in: what are the names of your classes and variables, what operating system you were using, if you had to change some header files, etc. Whenever possible (which is most of the time), please try to use images instead of text to explain concepts.\\
	Your project should include some amount of independent work, either by implementing a technique and showing its performance on some examples not included by the authors, or by doing some independent theoretical analysis.\\}

	\subsection{Adjustments for 2D}
	
the	polar issue, formulas different, etc...
  	
  	\subsection{Choice of the 1/N factor in the calculus of D}
  	
  	explain that it comes out of the blue ... \\
  	
  	\subsection{ non borné}
  	
  	la valeur du prst (monte carlo) obtenue telle qu'on le fait dépend complètement de l'unité de longueur qu'on utilise pour calculer les distance x,x'. Du coup on n'a vraiment aucune garantie sur les bornes du prst Monte Carlo, contrairement au prst complet...\\
  	
  	
  	\section{Results}
  		%TODO delete before saving
\textit{· 1.5-2 pages of results. Show (especially in, graphs, screenshots, etc.) the results that you have obtained. Comment on discrepancies (if any) with the results shown in the paper. Comment on whether you had to tweak parameters to get good results and, if so, how you picked them.\\}

	\section{Possible extensions}
	
\textit{· 0.5-1 pages of possible extensions. Can you suggest how the method can be improved? Can you suggest other application domains for your method?\\}

	\section*{Conclusion}
\addcontentsline{toc}{section}{Conclusion}

\addcontentsline{toc}{section}{References} 
\nocite{*}
\bibliographystyle{plain}
\bibliography{bibli}
		
\end{document}
